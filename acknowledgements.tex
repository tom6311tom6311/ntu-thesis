\begin{acknowledgementszh}
研究所第二年,撰寫論文的這段時光對我來說是求學生涯中特別踏實而難忘的一個篇章,這一年所經歷的種種求知、成長與蛻變讓向來倒頭就睡、不曾失眠的我,在口試的前一個晚上卻輾轉難眠了一整夜。當實驗、論文、簡報都已準備就緒並充分演練過,主宰我心的已非緊張不安的情緒,而是一種溫馨、喜悅與興奮摻雜而成的複雜情感:忙活了數個月,盡了一切該盡的努力後,我終於有時間回過頭來,想想那些生命中重要的人們,如何幫助我走過這段艱辛的研究旅程。

首先我要感謝人生中的導師 - 林宗男教授。在研究所兩年的時光中,老師幾乎每週都會找時間與我單獨討論研究進度,一路引導我定義問題、分析、提出解決方法、比較、設計實驗並觀察數據。我的研究主題並不是自己想出來或由老師指定的,而是在一次次的面談中逐漸由多個領域收斂出來的結果。謝謝老師因材施教,讓擅長實作多於理論的我能投入於這個應用層面較豐富的研究主題中;謝謝老師用心指導,每當我將修正完的投影片或新的實驗結果寄給老師,不管多忙碌,老師總會在隔日給我回覆或是找我前去討論;謝謝老師提供給我許多環境和資源,除了最好的經濟資助、最舒適的研究環境、最高級的設備外,我在研究之餘還有機會擔任電機系網多實驗、計算機程式、網路攻防實習等課程的助教,並且有許多機會親自準備教材、上台與同學分享我的所學;也謝謝老師持續提拔我走向下個階段。對於我和老師來說,我的碩士論文不只是一項研究,還是一個具有商業價值的核心技術,謝謝老師鼓勵我朝創業的方向前進,並且陪我持續思索可行的應用與解決方案。

接著我要感謝實驗室的夥伴:文于、煒傑、育維,我們從大二、大三就開始跟隨宗男教授,一起修課、做專題、當助教,一起分享許多生活瑣事以及研究上、學習上、工作上的心路歷程。儘管我們的研究主題相去甚遠,生活圈卻很接近。我們待在同一個實驗室中,各自忙著很不一樣的事情,卻都珍惜並讚嘆著彼此的能力。在畢業前兩三個月,我們一起為了論文與口試而忙碌,一起討論時程、論文格式以及各種需要注意的楣楣角角。在孤獨的研究之路上,能有這群相知相惜的朋友一路陪伴,真的是很爽快的事情!

除了老師和同學之外,我還要對同屬一個研究室的助理:小香姐、軒妤、柏盛、沛翰、佳宇、子建、合量,致上最深的感謝之意。每天早上我踏進研究室的那一刻開始,就充分地受到您們照顧,不論是安排與老師meeting、借討論室、彙整研究進度報告、預定餐點、報帳、助教時數表簽章、添購設備、口試安排、專利申請、門禁設定、設備維修、VM租借\dots 各種疑難雜症都在您們的協助下才得以順利解決。這兩年來我們每天在實驗室碰面,各自為不同任務而忙碌著,時而抬起頭聊聊天、時而相互照應、偶爾一起出遊,兩年的時間在您們的陪伴下很快就過了,但相信我們會成為一輩子的朋友,謝謝你們!

最後我要感謝我的家人,在口試的當天,我正要出門,爸爸開著貨車從對向經過,輕輕按聲喇叭,搖下車窗靦腆地對我說聲:“加油!” 這句簡短而熟悉的問候正是自有印象以來每當段考、基測、學測,爸爸都一定會對我說的一句話,十多年來始終如一!在繳交論文前兩三個月裡的每個深夜,我為了實驗數據而焦頭爛額時,媽媽總會在一旁追劇,時而偷瞥正在和螢幕上那一長串莫名程式碼奮鬥的我,陪我敖到半夜一兩點。在口試前一天,姐姐一如往常地命令我幫她買午餐、載她去搭火車、弟弟盧我幫他處理宿舍網路和民宿訂單的問題,他們一定是對我好,希望我平常心以對,嗯嗯一定是這樣的。

我要感謝所有在這段時間關心過我的家人、師長、朋友,給予我支持與寄託,讓我有機會全心全意地投入研究中,成為一個能夠發現、分析並解決問題的人。
\end{acknowledgementszh}
